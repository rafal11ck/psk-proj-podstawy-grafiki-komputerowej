% Created 2023-11-21 Tue 10:57
% Intended LaTeX compiler: pdflatex
\documentclass[11pt]{article}
\usepackage[utf8]{inputenc}
\usepackage[T1]{fontenc}
\usepackage{graphicx}
\usepackage{longtable}
\usepackage{wrapfig}
\usepackage{rotating}
\usepackage[normalem]{ulem}
\usepackage{amsmath}
\usepackage{amssymb}
\usepackage{capt-of}
\usepackage{hyperref}
\usepackage[margin=2cm]{geometry}
\author{Rafał Grot, Kamil Gunia, Piotr Górski}
\date{\today}
\title{Podstawy grafiki komputerwej 1 silnik do gier 2d}
\hypersetup{
 pdfauthor={Rafał Grot, Kamil Gunia, Piotr Górski},
 pdftitle={Podstawy grafiki komputerwej 1 silnik do gier 2d},
 pdfkeywords={},
 pdfsubject={},
 pdfcreator={Emacs 30.0.50 (Org mode 9.7)}, 
 pdflang={Polish}}
\begin{document}

\maketitle
\tableofcontents

\newpage
\section{Wymagania systemowe}
\label{sec:org7b065c9}
\begin{itemize}
\item SFML
\item cmake
\end{itemize}
\section{Użycie bilbioteki \texttt{engine}}
\label{sec:org2f6b2f0}
Aby mieć dostęp do silnika gry należy wykorzystać plik nagłówkowy \texttt{engine.hpp}..
\section{Wybrane klasy wykorzystywanie przez silnik.}
\label{sec:org67393b2}
\subsection{GameObject}
\label{sec:org3ef4a7c}
GameObject to klasa repezentująca obiekt w grze.
\subsection{Drawable}
\label{sec:org8ad77fe}
Drawable jest to abstrakcyjna klasa bazowa wszystkich klas, które można narysować.
\subsection{AnimatedObject}
\label{sec:org42dce6c}
Jest to klasa zawierająca metodę wirtualną \texttt{animate()}, należy ją przeciążyć w celu realizacji animacji.
\subsection{AnimatedSpriteSheet}
\label{sec:orgb54e728}
Klasa reprezentująca obiekty, które są animowane przy użyciu bitmap. Dziediczy po AnimatedObject.

\begin{itemize}
\item Obiekty tej klasy wczytują informacje o animacji z plików.

\item Konstruktor przyjmuje ścieżkę do katalogu, który musi zaweirać plik \texttt{config.txt}
\end{itemize}


Metadane animacji opisane w pliku konfiguracyjnym mają następujący format:

\begin{verbatim}
COMMAND
ARGS
\end{verbatim}

Bezpośrednio za linią z \texttt{COMMAND} musi znajdować się linia z argumentami oddzielanym spacją.
Linie które mają być traktowane jako komentarze zaczynają się znakiem "\#".
\subsubsection{Command}
\label{sec:orgc8e5692}

Prawidłowe opcje \texttt{COMMAND} to:
\begin{description}
\item[{\texttt{SPRITESHEET}}] Posiada jeden parametr -- scieżkę relatywną od katlagou, w którym znajduje się plik konfiguracyjny, do pliku zawierającego tablice spritów.

\item[{\texttt{ANIMATION}}] Oznacza ładowanie animacji, kolejne wywołania oznaczają nowe typy animacji. Nie przyjmuje parametrów.

\item[{\texttt{FRAME}}] Zawiera informacje o pojedyńczej klatce animacji.
Przyjmuje parametry oznaczające kolejno
\begin{itemize}
\item pozycje x lewego górnego rogu sprita.
\item pozycje y lewego górnego rogu sprita.
\item rozmiar w osi x sprita.
\item rozmiar w osi y sprita.
\item czas trwania klatki
\end{itemize}
\end{description}
\subsubsection{Przykład}
\label{sec:org2f7f7a6}
\begin{verbatim}
#Comment lines start with '#'
#SPRITESHEET should be followed with with relative path in the next line
#(from config file directory) to the spritesheet

#COMMENTS CAN NOT be in between INFO
#example:
#SPRITESHEET
##SOMECOMMENT
#filename.png
#
#is not allowed

#<path>
SPRITESHEET
spritesheet.png

#Indicates new animation
ANIMATION
#FRAME represents frame of animation
#it is followed by line containing
#<pos x> <pos y> <size x> <size y> <duration>
FRAME
0 0 100 100 0.5
FRAME
100 0 100 100 0 0.5
\end{verbatim}
\subsection{UpdateableObject}
\label{sec:org2df8fac}
Obiekty których stan logiczny się zmienia powinny przeciążać wirtualną metodoę \texttt{update} klasy \texttt{UpdateableObject}.
Silnik nie wywołuje tej metody, należy zadbać aby była ona wywoływana np. przy użyciu własnej pętli gry.
\section{Przykład użycia silnika}
\label{sec:orgb867cad}
Program wyświetlający animację.

\begin{verbatim}
#include "SFML/Graphics/Color.hpp"
#include "animatedSpriteSheet.hpp"
#include "engine.hpp"
#include <iostream>

namespace G {
std::string basePath = "resources/";
}; // namespace G

AnimatedSpriteSheet animation(G::basePath + "animation");

int main() {
  Engine::getInstance().setMaxFps(3).setResolution({1000, 1000}).buildWindow();

  animation.setPosition({300, 300});
  animation.setColor(sf::Color::Cyan);

  Engine::getInstance().add(&animation);

  Engine::getInstance().loop();
}
\end{verbatim}
\end{document}
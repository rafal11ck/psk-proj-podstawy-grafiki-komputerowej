% Created 2023-11-16 Thu 15:12
% Intended LaTeX compiler: pdflatex
\documentclass[11pt]{article}
\usepackage[utf8]{inputenc}
\usepackage[T1]{fontenc}
\usepackage{graphicx}
\usepackage{longtable}
\usepackage{wrapfig}
\usepackage{rotating}
\usepackage[normalem]{ulem}
\usepackage{amsmath}
\usepackage{amssymb}
\usepackage{capt-of}
\usepackage{hyperref}
\usepackage[margin=2cm]{geometry}
\author{Rafał Grot, Kamil Gunia, Piotr Górski}
\date{\today}
\title{Podstawy grafiki komputerwej 1 silnik do gier 2d}
\hypersetup{
 pdfauthor={Rafał Grot, Kamil Gunia, Piotr Górski},
 pdftitle={Podstawy grafiki komputerwej 1 silnik do gier 2d},
 pdfkeywords={},
 pdfsubject={},
 pdfcreator={Emacs 30.0.50 (Org mode 9.7)}, 
 pdflang={Polish}}
\begin{document}

\maketitle
\tableofcontents

\newpage
\section{{\bfseries\sffamily DONE} Wymagania systemowe}
\label{sec:orgb63ae14}
\begin{itemize}
\item SFML
\item cmake
\end{itemize}
\section{{\bfseries\sffamily TODO} Jak używać silnika}
\label{sec:orge06f6c1}
\subsection{{\bfseries\sffamily TODO} użycie bilbioteki \texttt{engine}}
\label{sec:org6b00b3a}
\subsection{{\bfseries\sffamily TODO} Dodawnie obiektów do silnika}
\label{sec:org6e73772}
\subsubsection{{\bfseries\sffamily TODO} GameObject}
\label{sec:org9174d34}
GameObject to klasa repezentująca obiekt w grze.
\subsubsection{{\bfseries\sffamily TODO} Drawable}
\label{sec:org9cebd53}
Drawable jest to abstrakcyjna klasa bazowa wszystkich klas, które można narysować
\subsubsection{{\bfseries\sffamily TODO} AnimatedObject}
\label{sec:orga1f3cea}
\end{document}
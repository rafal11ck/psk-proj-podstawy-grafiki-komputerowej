% Created 2023-12-01 Fri 14:47
% Intended LaTeX compiler: pdflatex
\documentclass[11pt]{article}
\usepackage[utf8]{inputenc}
\usepackage[T1]{fontenc}
\usepackage{graphicx}
\usepackage{longtable}
\usepackage{wrapfig}
\usepackage{rotating}
\usepackage[normalem]{ulem}
\usepackage{amsmath}
\usepackage{amssymb}
\usepackage{capt-of}
\usepackage{hyperref}
\usepackage[margin=2cm]{geometry}
\author{Rafał Grot, Kamil Gunia, Piotr Górski}
\date{\today}
\title{Podstawy grafiki komputerwej 1 silnik do gier 2d}
\hypersetup{
 pdfauthor={Rafał Grot, Kamil Gunia, Piotr Górski},
 pdftitle={Podstawy grafiki komputerwej 1 silnik do gier 2d},
 pdfkeywords={},
 pdfsubject={},
 pdfcreator={Emacs 30.0.50 (Org mode 9.7)}, 
 pdflang={Polish}}
\begin{document}

\maketitle
\newpage
\section{Wymagania systemowe}
\label{sec:org8d8e0e6}
\begin{itemize}
\item SFML
\item cmake
\end{itemize}
\section{Z wymagań}
\label{sec:orged4aa2d}
\subsection{Obsługa klawiatury i myszy}
\label{sec:org33c34ec}
Wykorzystaj \texttt{setEventHandler()}. Po szczegóły zajrzyj do dokumentacji.

\begin{itemize}
\item Przykład
\end{itemize}
\begin{verbatim}
#include "engine.hpp"
#include <iostream>

int main() {
  Engine &eng = Engine::getInstance().setWindowTitle("dev").buildWindow();

  eng.setEventHandler(
      Engine::Event::MouseButtonPressed, [](const Engine::Event &ev) {
        std::cout << "Custom event handler Mouse button press "
                  << ev.mouseButton.button << '\t' << ev.mouseButton.x << '\t'
                  << ev.mouseButton.y << "\n";
      });

  eng.loop();
}
\end{verbatim}
\subsection{Obsługa współrzędnych (Point2D)}
\label{sec:org79b33ff}
Klasa \texttt{Point2d}. Po szczegóły zajrzyj do dokumentacji.
\subsection{Rysowanie prymitywów}
\label{sec:org7402ac5}
Przy użyciu klass z bilbioteki SFML oraz metod \texttt{Engine::add()} i \texttt{Engine::remove()}.

Szeczóły w dokumentacji technicznej oraz dokumentacji SFML.
\subsubsection{Przykład}
\label{sec:org3c0a5df}
\begin{verbatim}
#include "SFML/Graphics/CircleShape.hpp"
#include "engine.hpp"
#include <iostream>

int main() {
  Engine &eng = Engine::getInstance().setWindowTitle("dev").buildWindow();

  sf::CircleShape *circle = new sf::CircleShape(100);

  circle->setPosition(100, 100);

  eng.add(circle);

  eng.loop();
}
\end{verbatim}
\subsection{Wypełnianie prymitywów kolorem}
\label{sec:orgbac75d8}
Przy użyciu metod obiektów z biblioteki SFML.
\subsubsection{Przykład}
\label{sec:org7046b3b}
\begin{verbatim}
#include "SFML/Graphics/CircleShape.hpp"
#include "SFML/Graphics/Color.hpp"
#include "engine.hpp"
#include <cstdlib>
#include <iostream>
#include <random>

sf::CircleShape *circle = new sf::CircleShape(100);

void customLoop() {
  // count to second
  static float dt{};
  dt += Engine::getInstance().getLastFrameDuration().asSeconds();

  // Do not change color if less then scond elapsed.
  if (dt < 1)
    return;
  dt = 0;

  // Set color based on random RGB values
  sf::Color randomColor{static_cast<sf::Uint8>(rand() % 265),
                        static_cast<sf::Uint8>(rand() % 256),
                        static_cast<sf::Uint8>(rand() % 256)};
  // Set fill color
  circle->setFillColor(randomColor);
}

int main() {
  // initalize random number generator in order to randomize colors.
  srand(time(0));

  Engine &eng =
      Engine::getInstance().setMaxFps(75).setWindowTitle("dev").buildWindow();

  circle->setPosition(100, 100);

  // set custom loop function to change color.
  eng.setLoopFunction(customLoop);

  eng.add(circle);

  eng.loop();
}
\end{verbatim}
\subsection{Przekształcenia geometryczne}
\label{sec:org911a2e1}
Przy użyciu SFML.
\subsubsection{Przykład}
\label{sec:org6658196}

\begin{verbatim}
#include "SFML/Graphics/RectangleShape.hpp"
#include "engine.hpp"
#include <iostream>

// This is example of transformation usate.
// Consider using AnimatedObject class if you want to do animations based on
// time elapsed

sf::RectangleShape *rect = new sf::RectangleShape({100, 100});

void customLoop() {
  static float dt{};
  dt += Engine::getInstance().getLastFrameDuration().asSeconds();

  // after some time execute step
  if (dt < 0.3)
    return;

  dt = 0;
  static int step{};

  // each second do one of following
  switch (step) {
    // step 0 move right by 100
  case 0:
    rect->move(100, 0);
    ++step;
    break;
  case 1:
    // make rectangle 2 times bigger
    rect->scale(2, 2);
    ++step;
    break;
  case 2:
    // rotate it
    // Check SFML documentation
    rect->setOrigin(rect->getSize().x / 2, rect->getSize().y / 2);

    // rotate by 45 degress
    rect->rotate(45);
    ++step;
    break;

  case 3:
    // Shrink it
    rect->scale(0.5, 0.5);
    ++step;
    break;
  case 4:
    // Move it to starting position
    rect->move(-100, 0);
    ++step;
    break;

  default:
    // reset step
    step = 0;
  }
}

int main() {
  Engine &eng = Engine::getInstance().setWindowTitle("dev").buildWindow();

  rect->setPosition(100, 300);

  eng.add(rect);

  eng.setLoopFunction(customLoop);

  eng.loop();
}
\end{verbatim}
\subsection{Obsługa bitmap}
\label{sec:orgcdbd3b8}
Przy użyciu biblioteki SFML tak jak z primitywami.
\subsubsection{Przykład (\texttt{test8})}
\label{sec:orge72370f}

\begin{verbatim}
#include "SFML/Graphics/Sprite.hpp"
#include "engine.hpp"
#include <iostream>

sf::Sprite *sprite = new sf::Sprite{};

int main() {
  Engine &eng = Engine::getInstance().setWindowTitle("dev").buildWindow();

  sf::Texture texture;

  // texture file paths are relative to executable file, not source file.
  if (!texture.loadFromFile("textureFile.png", sf::IntRect(0, 0, 200, 200))) {
    std::cout << "Could not find texture file";
  }

  sprite->setTexture(texture);
  eng.add(sprite);

  eng.loop();
}
\end{verbatim}
\subsection{Animowanie bitmap}
\label{sec:org6680f52}
Przy użyciu klasy \texttt{AnimatedSpriteSheet}. Szecgóły w dokumetnacji technicznej oraz w dalszej częsci sprawozdania.

\begin{verbatim}
#include "SFML/Graphics/Color.hpp"
#include "animatedSpriteSheet.hpp"
#include "engine.hpp"
#include <iostream>

namespace G {
std::string basePath = "resources/";
}; // namespace G

AnimatedSpriteSheet animation(G::basePath + "animation");

int main() {
  Engine::getInstance().setMaxFps(3).setResolution({1000, 1000}).buildWindow();

  animation.setPosition({300, 300});
  animation.setColor(sf::Color::Cyan);

  Engine::getInstance().add(&animation);

  Engine::getInstance().loop();
}
\end{verbatim}
\section{Użycie biblioteki \texttt{engine}}
\label{sec:org6940c44}
Aby mieć dostęp do silnika gry należy wykorzystać plik nagłówkowy \texttt{engine.hpp} oraz zlinkować bibliotekę \texttt{engine}
Klasa Engine jest singletonem.
Zaleca się zaposanie z dokumentacją techniczną.
Po ustawieniu parametrów należy wywołać metodę \texttt{buildWindow()}.
Następnie o ile potrzebna wykonać inicjalizację pętli użytkownika.
Na koniec należy wywołać metodę silnika \texttt{loop()}.

\begin{verbatim}
int main(){
Engine::getInstance().setMaxFps(3).setResolution({1000, 1000}).buildWindow();

// ...

Engine::getInstance().loop();
}
\end{verbatim}
\section{Wybrane klasy wykorzystywanie przez silnik.}
\label{sec:org8847283}
\subsection{Prymitywy}
\label{sec:orgacdf54f}
Przy wykorzystaniu Klasa z biblioteki SFML stworzyć obiekty dziedziczące po \texttt{sf::Drawable} oraz dodać je do silnika za pomocą metody silnika \texttt{Engine::add()}.
\subsection{GameObject}
\label{sec:orgbbacc67}
GameObject to klasa reprezentująca obiekt w grze.
\subsection{Drawable}
\label{sec:orgbfbd0f7}
Drawable jest to abstrakcyjna klasa bazowa wszystkich klas, które można narysować.
Dokładnie jest to alias do \texttt{sf::Drawable}.
\subsection{AnimatedObject}
\label{sec:org0d80b3e}
Jest to klasa zawierająca metodę wirtualną \texttt{animate()}, należy ją przeciążyć w celu realizacji animacji.
Powinna być użyta w dla animacji jeśli inna klasa obsługująca animacje nie jest odpowiednia.
\subsection{AnimatedSpriteSheet}
\label{sec:orgcc1f4b2}
Klasa reprezentująca obiekty, które są animowane przy użyciu bitmap. Dziedziczy po AnimatedObject.

\begin{itemize}
\item Obiekty tej klasy wczytują informacje o animacji z plików.

\item Konstruktor przyjmuje ścieżkę do katalogu, który musi zawierać plik \texttt{config.txt}
\end{itemize}

Metadane animacji opisane w pliku konfiguracyjnym mają następujący format:

\begin{verbatim}
COMMAND
ARGS
\end{verbatim}

Bezpośrednio za linią z \texttt{COMMAND} musi znajdować się linia z argumentami oddzielanym spacją.
Linie które mają być traktowane jako komentarze zaczynają się znakiem "\#".
\subsubsection{Command}
\label{sec:org91d437f}

Prawidłowe kommendy \texttt{COMMAND} to:
\begin{description}
\item[{\texttt{SPRITESHEET}}] Posiada jeden parametr -- ścieżkę relatywną od katalogu, w którym znajduje się plik konfiguracyjny, do pliku zawierającego tablice spirtów.

\item[{\texttt{ANIMATION}}] Oznacza ładowanie animacji, kolejne wywołania oznaczają nowe typy animacji. Nie przyjmuje parametrów.

\item[{\texttt{FRAME}}] Zawiera informacje o pojedynczej klatce animacji.
Przyjmuje parametry oznaczające kolejno
\begin{itemize}
\item pozycje x lewego górnego rogu sprita.
\item pozycje y lewego górnego rogu sprita.
\item rozmiar w osi x sprita.
\item rozmiar w osi y sprita.
\item czas trwania klatki
\end{itemize}
\end{description}
\subsubsection{Przykład}
\label{sec:org79e4ddf}
\begin{verbatim}
#Comment lines start with '#'
#SPRITESHEET should be followed with with relative path in the next line
#(from config file directory) to the spritesheet

#COMMENTS CAN NOT be in between INFO
#example:
#SPRITESHEET
##SOMECOMMENT
#filename.png
#
#is not allowed

#<path>
SPRITESHEET
spritesheet.png

#Indicates new animation
ANIMATION
#FRAME represents frame of animation
#it is followed by line containing
#<pos x> <pos y> <size x> <size y> <duration>
FRAME
0 0 100 100 0.5
FRAME
100 0 100 100 0 0.5
\end{verbatim}
\subsection{UpdateableObject}
\label{sec:orgb68f463}
Obiekty których stan logiczny się zmienia powinny przeciążać wirtualną metodę \texttt{update} klasy \texttt{UpdateableObject}.
Silnik nie wywołuje tej metody, należy zadbać aby była ona wywoływana np. przy użyciu własnej pętli gry.
\section{Przykład użycia silnika}
\label{sec:orgb901082}
Program wyświetlający animację.

\begin{verbatim}
#include "SFML/Graphics/Color.hpp"
#include "animatedSpriteSheet.hpp"
#include "engine.hpp"
#include <iostream>

namespace G {
std::string basePath = "resources/";
}; // namespace G

AnimatedSpriteSheet animation(G::basePath + "animation");

int main() {
  Engine::getInstance().setMaxFps(3).setResolution({1000, 1000}).buildWindow();

  animation.setPosition({300, 300});
  animation.setColor(sf::Color::Cyan);

  Engine::getInstance().add(&animation);

  Engine::getInstance().loop();
}
\end{verbatim}
\end{document}
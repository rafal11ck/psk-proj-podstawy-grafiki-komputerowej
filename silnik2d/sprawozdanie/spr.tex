% Created 2023-11-30 Thu 16:57
% Intended LaTeX compiler: pdflatex
\documentclass[11pt]{article}
\usepackage[utf8]{inputenc}
\usepackage[T1]{fontenc}
\usepackage{graphicx}
\usepackage{longtable}
\usepackage{wrapfig}
\usepackage{rotating}
\usepackage[normalem]{ulem}
\usepackage{amsmath}
\usepackage{amssymb}
\usepackage{capt-of}
\usepackage{hyperref}
\usepackage[margin=2cm]{geometry}
\author{Rafał Grot, Kamil Gunia, Piotr Górski}
\date{\today}
\title{Podstawy grafiki komputerwej 1 silnik do gier 2d}
\hypersetup{
 pdfauthor={Rafał Grot, Kamil Gunia, Piotr Górski},
 pdftitle={Podstawy grafiki komputerwej 1 silnik do gier 2d},
 pdfkeywords={},
 pdfsubject={},
 pdfcreator={Emacs 30.0.50 (Org mode 9.7)}, 
 pdflang={Polish}}
\begin{document}

\maketitle
\newpage
\section{Wymagania systemowe}
\label{sec:orga4224dc}
\begin{itemize}
\item SFML
\item cmake
\end{itemize}
\section{Użycie biblioteki \texttt{engine}}
\label{sec:org4590a0b}
Aby mieć dostęp do silnika gry należy wykorzystać plik nagłówkowy \texttt{engine.hpp} oraz zlinkować bibliotekę \texttt{engine}
Klasa Engine jest singletonem.
Zaleca się zaposanie z dokumentacją techniczną.
Po ustawieniu parametrów należy wywołać metodę \texttt{buildWindow()}.
Następnie o ile potrzebna wykonać inicjalizację pętli użytkownika.
Na koniec należy wywołać metodę silnika \texttt{loop()}.

\begin{verbatim}
int main(){
Engine::getInstance().setMaxFps(3).setResolution({1000, 1000}).buildWindow();

// ...

Engine::getInstance().loop();
}
\end{verbatim}
\section{Wybrane klasy wykorzystywanie przez silnik.}
\label{sec:org1db87f1}
\subsection{Prymitywy}
\label{sec:org65b32db}
Przy wykorzystaniu Klasa z biblioteki SFML stworzyć obiekty dziedziczące po \texttt{sf::Drawable} oraz dodać je do silnika za pomocą metody silnika \texttt{Engine::add()}.
\subsection{GameObject}
\label{sec:org0af1d92}
GameObject to klasa reprezentująca obiekt w grze.
\subsection{Drawable}
\label{sec:orgcf0a222}
Drawable jest to abstrakcyjna klasa bazowa wszystkich klas, które można narysować.
Dokładnie jest to alias do \texttt{sf::Drawable}.
\subsection{AnimatedObject}
\label{sec:org9a2b6fe}
Jest to klasa zawierająca metodę wirtualną \texttt{animate()}, należy ją przeciążyć w celu realizacji animacji.
Powinna być użyta w dla animacji jeśli inna klasa obsługująca animacje nie jest odpowiednia.
\subsection{AnimatedSpriteSheet}
\label{sec:orgc117c95}
Klasa reprezentująca obiekty, które są animowane przy użyciu bitmap. Dziedziczy po AnimatedObject.

\begin{itemize}
\item Obiekty tej klasy wczytują informacje o animacji z plików.

\item Konstruktor przyjmuje ścieżkę do katalogu, który musi zawierać plik \texttt{config.txt}
\end{itemize}

Metadane animacji opisane w pliku konfiguracyjnym mają następujący format:

\begin{verbatim}
COMMAND
ARGS
\end{verbatim}

Bezpośrednio za linią z \texttt{COMMAND} musi znajdować się linia z argumentami oddzielanym spacją.
Linie które mają być traktowane jako komentarze zaczynają się znakiem "\#".
\subsubsection{Command}
\label{sec:orga941f1e}

Prawidłowe kommendy \texttt{COMMAND} to:
\begin{description}
\item[{\texttt{SPRITESHEET}}] Posiada jeden parametr -- ścieżkę relatywną od katalogu, w którym znajduje się plik konfiguracyjny, do pliku zawierającego tablice spirtów.

\item[{\texttt{ANIMATION}}] Oznacza ładowanie animacji, kolejne wywołania oznaczają nowe typy animacji. Nie przyjmuje parametrów.

\item[{\texttt{FRAME}}] Zawiera informacje o pojedynczej klatce animacji.
Przyjmuje parametry oznaczające kolejno
\begin{itemize}
\item pozycje x lewego górnego rogu sprita.
\item pozycje y lewego górnego rogu sprita.
\item rozmiar w osi x sprita.
\item rozmiar w osi y sprita.
\item czas trwania klatki
\end{itemize}
\end{description}
\subsubsection{Przykład}
\label{sec:orga2a986e}
\begin{verbatim}
#Comment lines start with '#'
#SPRITESHEET should be followed with with relative path in the next line
#(from config file directory) to the spritesheet

#COMMENTS CAN NOT be in between INFO
#example:
#SPRITESHEET
##SOMECOMMENT
#filename.png
#
#is not allowed

#<path>
SPRITESHEET
spritesheet.png

#Indicates new animation
ANIMATION
#FRAME represents frame of animation
#it is followed by line containing
#<pos x> <pos y> <size x> <size y> <duration>
FRAME
0 0 100 100 0.5
FRAME
100 0 100 100 0 0.5
\end{verbatim}
\subsection{UpdateableObject}
\label{sec:orgba3956d}
Obiekty których stan logiczny się zmienia powinny przeciążać wirtualną metodę \texttt{update} klasy \texttt{UpdateableObject}.
Silnik nie wywołuje tej metody, należy zadbać aby była ona wywoływana np. przy użyciu własnej pętli gry.
\section{Przykład użycia silnika}
\label{sec:org5220c91}
Program wyświetlający animację.

\begin{verbatim}
#include "SFML/Graphics/Color.hpp"
#include "animatedSpriteSheet.hpp"
#include "engine.hpp"
#include <iostream>

namespace G {
std::string basePath = "resources/";
}; // namespace G

AnimatedSpriteSheet animation(G::basePath + "animation");

int main() {
  Engine::getInstance().setMaxFps(3).setResolution({1000, 1000}).buildWindow();

  animation.setPosition({300, 300});
  animation.setColor(sf::Color::Cyan);

  Engine::getInstance().add(&animation);

  Engine::getInstance().loop();
}
\end{verbatim}
\section{Z wymagań}
\label{sec:orgb006597}
\subsection{Obsługa klawiatury i myszy}
\label{sec:org43269a6}
Wykorzystaj \texttt{setEventHandler()}. Po szczegóły zajrzyj do dokumentacji.

\begin{verbatim}
#include "engine.hpp"
#include <iostream>

int main() {
  Engine &eng = Engine::getInstance().setWindowTitle("dev").buildWindow();

  eng.setEventHandler(
      Engine::Event::MouseButtonPressed, [](const Engine::Event &ev) {
        std::cout << "Custom event handler Mouse button press "
                  << ev.mouseButton.button << '\t' << ev.mouseButton.x << '\t'
                  << ev.mouseButton.y << "\n";
      });

  eng.loop();
}
\end{verbatim}
\subsection{Obsługa współrzędnych (Point2D)}
\label{sec:org9666499}
Klasa \texttt{Point2d}. Po szczegóły zajrzyj do dokumentacji.
\subsection{Rysowanie prymitywów}
\label{sec:orgdeb4bf1}
Przy użyciu klass z bilbioteki SFML oraz metod \texttt{Engine::add()} i \texttt{Engine::remove()}.
Szeczóły w dokumentacji technicznej oraz dokumentacji SFML.
\subsection{Wypełnianie prymitywów kolorem}
\label{sec:orgfabc6a9}
Przy użyciu metod obiektów z biblioteki SFML.
\subsection{Przekształcenia geometryczne}
\label{sec:org52973b8}
Przy użyciu SFML.
\subsection{Obsługa bitmap}
\label{sec:org998f764}
Przy użyciu biblioteki SFML tak jak z primitywami.
\subsection{Animowanie bitmap}
\label{sec:orgd573061}
Przy użyciu klasy \texttt{AnimatedSpriteSheet}. Szecgóły w dokumetnacji technicznej.
\end{document}